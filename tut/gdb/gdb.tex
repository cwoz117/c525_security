\documentclass{article}
\usepackage[hidelinks]{hyperref}
\usepackage{amsthm}
\usepackage{listings}

\author{Chris Wozniak}

\begin{document}
\section*{GDB Debugger}
	The purpose of GDB is to allow you to see what is going on ``inside" another program while it
	executes, or what went on the moment it crashed. It is primarily for C and C++ and can be opened with 
	\textit{gdb} and closed with \textit{quit, or q}\\

\section*{Useful extraneous features}
	\begin{itemize}
		\item \textbf{Tab Complete} commands will run the first unambiguous command. Where there is
			ambiguity, it will list out the possible commands. Some commands (like step) are hard coded
			to work with "s" even though other commands start with s making this ambiguous.
		\item \textbf{Shell Access} can be done through gdb, so we do not have to exit gdb. Simply
			prepend the word "shell" to any command line action.
		\item \textbf{Logging} output can be set to a file, which defaults to "gdb.txt" this is achieved through
			a few methods:
			\begin{itemize}
				\item \textit{set logging $<$on/off$>$} to activate.
				\item \textit{set logging file $<$filename$>$}
				\item \textit{show logging}
			\end{itemize}
		\item \textbf{Bookmarks} gdb is able to save a ``snapshot" of a programs state in certain machines, called a 
			checkpoint. Returning to the checkmark undoes everything that happened since the save. Each checkpoint is
			given an integer ID value, to organize which checkpoints we have.
			\begin{itemize}
				\item \textit{checkpoint} to save the state. Takes no arguments
				\item \textit{info checkpoints} to view a list of checkpoints, and some status info.
				\item \textit{restart $<$checkpoint\_ID$>$} to restore the program state.
				\item \textit{delete checkpoint $<$checkpoint\_ID$>$}
			\end{itemize}
		\item \textbf{Compiler support} GDB prefers to have any code compiled with the \textit{-g} function, which 
			generates debugging info and stores it in the object file. It can be done without, but opens up more 
			features if it does. \textit{gcc} allows the \textit{-g} flag with the \textit{-o} flag to allow for debugging 
			optimized code, and GDB reccommends \textbf{always} leaving the \textit{-g} flag on.
	\end{itemize}
\newpage
\section*{Running Programs with GDB}
	\begin{itemize}
		\item \textbf{gdb start} consist of a program, and a core file or process ID. 
			\begin{itemize}
				\item \textit{gdb program.o core\_or\_PID} command. Note that the args are optional, and the 
					most common is to drop the 3rd argument.
				\item \textit{run} starts your program, if you are working on an embedded device you should 
					use \textit{continue} instead, since your program may not have a "start" function like main.
				\item \textit{start} is the quivilent of setting a temporary breakpoint at main, then running the program.
				\item \textit{attach $<$pid$>$/detach} connects gdb to a running process, or detaches it after it finishes.
				\item \textit{kill} kills the child process.
			\end{itemize}
		\item \textbf{System Enviroment} consists of a set of enviroment variables and their values. They 
			conventionally record your username, home directory, terminal type, and search path for programs to 
			run. Normally they take everything from the main gdb process, but it may be useful to change them.
			\begin{itemize}
				\item \textit{path $<$directory$>$} will add the directory to the front of the PATH enviroment variable. 
					This does not change the path that GDB uses, only the inferior process (running program).
				\item \textit{show paths} will display a list of search paths.
				\item \textit{show enviroment} will print the print the enviroment variable ``varname" to be given to 
					your program. If varname is not supplied, it will print all names and values of all enviroment 
					variables given to your program.
				\item \textit{set env varname $=$value} sets value in the varname enviroment.
				\item \textit{target remote} allows us to perform debugging on remote targets, otherwise the program will be
					created as a native process, under gdb on your local machine.
				\item \textit{Arguments} can be passed to the \textit{run} command which will in turn pass them to 
					the inferior process.
			\end{itemize}
		\item \textbf{Working Directory} is inherited from gdb, but can be changed with the \textit{cd} command, and we
			can check the processes working directory with \textit{pwd}.
		\item \textbf{I/O} \textit{info terminal} displays info recorded by gdb about the terminal modes your 
			program is running. We can redirect a programs I/O using shell redirection: \textit{run $>$ outfile} 
			or use the \textit{tty} command to get more leverage over I/O.
	\end{itemize}
	GDB does offer support for running multiple programs, including forked processes and multithreading, but this is much
	more complex and RTFM is highly reccommended.
\newpage
\section*{Breakpoints}
	The principal purposes of using debuggers are so that you can stop your program before it terminates. If your program
	runs into trouble we can investigate and find out why. \textit{info program} can tell us specifics about the inferior process.
	All breakpoints are given an integer name, and can be \textit{enabled} or \textit{disabled}
	\begin{itemize}
		\item \textbf{Breakpoint} makes your program stop whenever a certain point in the program is reached. For each
			breakpoint, we can add conditions to control in finer detail whether a program stops.
			\begin{itemize}
				\item \textit{break $<$location$>$} Sets a breakpoint at location. Which can be a function name, a
					line number, or an address of an instruction. There are many ways to set the location, RTFM for more.
				\item \textit{break ... if $<$condition$>$} where condition can be a regex, or expression.
				\item \textit{info breakpoints} will print a table of all breakpoints with basic status info.
			\end{itemize}
		\item \textbf{Watchpoint} is a special breakpoint which stops your program when the value of an expression changes.
			This could be the value of a variable, or one or more operators combined: '$a + b$'. These are sometimes
			called ``data breakpoints"
			\begin{itemize}
				\item \textit{watch [location, regex, expression, threadnum, or mask] }
			\end{itemize}
		\item \textbf{Catchpoint} is another breakpoint that stops when a certain event occurs, such as throwing an exception
			(in C++) and or the loading of a library.

	\end{itemize}


\end{document}
