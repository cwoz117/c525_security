\documentclass{article}
\usepackage[hidelinks]{hyperref}
\usepackage{amsthm}
\usepackage{listings}

\author{Chris Wozniak}

\begin{document}
\section*{GDB Debugger}
	The purpose of GDB is to allow you to see what is going on ``inside" another program while it
	executes, or what went on the moment it crashed. It is primarily for C and C++\\
	
	GDB can do 4 main kinds of things:
	\begin{enumerate}
		\item Start your program, can be passed arguments.
		\item Make your program stop on specified conditions.
		\item Examine what has happened when your program stopped
		\item Change things in your program, so you can experiment with correcting
			the effects of one bug, to learn about others.
	\end{enumerate}
	The first key thing to point out is that GDB will tab complete commands, and where there is
	ambiguity, it will list out the possible commands. Here we will display the full command
	to be clear, but other examples may include shorthand.

\section*{Running GDB}
	To run gdb, simply use the command "gdb outfile.o" which will load the program bytecode
	to prepare it to execute. We can also just run gdb, and then use the load command to load
	whichever file we specify.\\

	Once we have the program loaded, we can run the program by calling the ``run" command, which
	will make the program execute and complete displaying any I/O. If your program requires
	any arguments, add them after run as if it were normal execution.

\section*{Breakpoints}
	We can set a breakpoint to watch out for a particular string in the code. For example, 
	we could call:
	\begin{lstlisting}
	(gdb) break main
	\end{lstlisting}
	To set a break point when the main function is called. This will halt program execution		
	to allow us to run gdb commands on our program during runtime.

\section*{next vs. step}
	The command ``next" can be used to advance execution to the next line of the current function.
	``Current Function" is the key word here, if we would like to advance to the next command
	in \textbf{any} subroutine, we must instead use the ``step" command. This allows us to pass
	over functions we know are okay, or drill down to follow the line of execution explicitly.




\end{document}
