\documentclass{report}
\usepackage[hidelinks]{hyperref}
\usepackage{amsthm}

\theoremstyle{definition}
\newtheorem*{examp}{Ex}

\title{Computer Security Textbook Notes}
\author{Chris Wozniak}

\begin{document}
\maketitle
\tableofcontents
\chapter{History of Computer Security}
	Computer Security challenges change with the times, and will continue to change. The first chapter traces the history of 
	computer security, placing the security mechanisms into perspective of the IT landscape they were developed for.
	\section{Dawn of Computer Security}
		New security Challenges arise when new, or old technologies are put to new use. With multi-user systems emerging in
		the 1960's computers needed mechanisms to protect itself against its users, and users from each other. However the
		concept of security being a research field in its own right did not start until the early 1970's and has only become a 
		larger problem today.
		\subsection{1970's - Mainframes}
			The introduction of mainframes to large business/governments came with two applications that required security
			measures:
			\begin{itemize}
				\item Defence sector saw the benefit of computers, but classified information had to be processed securely.
					This led to the multi-level user system being built into the OS.
				\item Governments processing unclassified but sensitive data, like citizens personal data. With multiple users
					working on large systems, there needed to be measures of security to prevent unauthorized access of
					files that should be kept private.
			\end{itemize}
			Access control mechanisms in the OS had to support multi-user security, only allowing sharing of data when it was 
			explicitly allowed. Encryption was seen to provide the most comprehensive protection for data stored in 
			memory/backup media. This led to DES, and Public Key Cryptography being published in 1976.

		\subsection{1980's - Personal Computers}
		\subsection{1990's - Internet}
		\subsection{2000's - Web}

\chapter {Managing Security}
	






\end{document}