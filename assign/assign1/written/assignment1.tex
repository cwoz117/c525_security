%%%%%%%%%%%%%%%%%%%%%%%%%%%%%%%%%%%%%%%%%
% Programming/Coding Assignment
% LaTeX Template
%
% This template has been downloaded from:
% http://www.latextemplates.com
%
% Original author:
% Ted Pavlic (http://www.tedpavlic.com)
%
% Note:
% to fill the template out. These commands should all be removed when 
% writing assignment content.
%
% This template uses a Perl script as an example snippet of code, most other
% languages are also usable. Configure them in the "CODE INCLUSION 
% CONFIGURATION" section.
%
%%%%%%%%%%%%%%%%%%%%%%%%%%%%%%%%%%%%%%%%%

%----------------------------------------------------------------------------------------
%	PACKAGES AND OTHER DOCUMENT CONFIGURATIONS
%----------------------------------------------------------------------------------------

\documentclass{article}

\usepackage{fancyhdr} % Required for custom headers
\usepackage{extramarks} % Required for headers and footers
\usepackage[usenames,dvipsnames]{color} % Required for custom colors
\usepackage{graphicx} % Required to insert images
\usepackage{listings} % Required for insertion of code
\usepackage{courier} % Required for the courier font

% Margins
\topmargin=-0.45in
\evensidemargin=0in
\oddsidemargin=0in
\textwidth=6.5in
\textheight=9.0in
\headsep=0.25in

\linespread{1.1} % Line spacing

\setlength\parindent{0pt} % Removes all indentation from paragraphs

%----------------------------------------------------------------------------------------
%	CODE INCLUSION CONFIGURATION
%----------------------------------------------------------------------------------------

\definecolor{MyDarkGreen}{rgb}{0.0,0.4,0.0} % This is the color used for comments
\lstloadlanguages{Perl} % Load Perl syntax for listings, for a list of other languages supported see: ftp://ftp.tex.ac.uk/tex-archive/macros/latex/contrib/listings/listings.pdf
\lstset{language=c,
        frame=single, % Single frame around code
        basicstyle=\small\ttfamily, % Use small true type font
        identifierstyle=, % Nothing special about identifiers                                         
        showstringspaces=false, % Don't put marks in string spaces
        tabsize=5, % 5 spaces per tab
        numbers=left, % Line numbers on left
        firstnumber=1, % Line numbers start with line 1
        numberstyle=\tiny\color{Blue}, % Line numbers are blue and small
}

% Creates a new command to include a perl script, the first parameter is the filename of the script (without .pl), the second parameter is the caption
\newcommand{\perlscript}[2]{
\begin{itemize}
\item[]\lstinputlisting[caption=#2,label=#1]{#1.pl}
\end{itemize}
}

%----------------------------------------------------------------------------------------
%	DOCUMENT STRUCTURE COMMANDS
%	Skip this unless you know what you're doing
%----------------------------------------------------------------------------------------

% Header and footer for when a page split occurs within a problem environment
\newcommand{\enterProblemHeader}[1]{
\nobreak\extramarks{#1}{#1 continued on next page\ldots}\nobreak
\nobreak\extramarks{#1 (continued)}{#1 continued on next page\ldots}\nobreak
}

% Header and footer for when a page split occurs between problem environments
\newcommand{\exitProblemHeader}[1]{
\nobreak\extramarks{#1 (continued)}{#1 continued on next page\ldots}\nobreak
\nobreak\extramarks{#1}{}\nobreak
}

\setcounter{secnumdepth}{0} % Removes default section numbers
\newcounter{homeworkProblemCounter} % Creates a counter to keep track of the number of problems

\newcommand{\homeworkProblemName}{}
\newenvironment{homeworkProblem}[1][Problem \arabic{homeworkProblemCounter}]{ % Makes a new environment called homeworkProblem which takes 1 argument (custom name) but the default is "Problem #"
\stepcounter{homeworkProblemCounter} % Increase counter for number of problems
\renewcommand{\homeworkProblemName}{#1} % Assign \homeworkProblemName the name of the problem
\section{\homeworkProblemName} % Make a section in the document with the custom problem count
\enterProblemHeader{\homeworkProblemName} % Header and footer within the environment
}{
\exitProblemHeader{\homeworkProblemName} % Header and footer after the environment
}

\newcommand{\problemAnswer}[1]{ % Defines the problem answer command with the content as the only argument
\noindent\framebox[\columnwidth][c]{\begin{minipage}{0.98\columnwidth}#1\end{minipage}} % Makes the box around the problem answer and puts the content inside
}

\newcommand{\homeworkSectionName}{}
\newenvironment{homeworkSection}[1]{ % New environment for sections within homework problems, takes 1 argument - the name of the section
\renewcommand{\homeworkSectionName}{#1} % Assign \homeworkSectionName to the name of the section from the environment argument
\subsection{\homeworkSectionName} % Make a subsection with the custom name of the subsection
\enterProblemHeader{\homeworkProblemName\ [\homeworkSectionName]} % Header and footer within the environment
}{
\enterProblemHeader{\homeworkProblemName} % Header and footer after the environment
}

%----------------------------------------------------------------------------------------
%	NAME AND CLASS SECTION
%----------------------------------------------------------------------------------------

\newcommand{\hmwkTitle}{Assignment\ \#1} 					% Assignment title
\newcommand{\hmwkDueDate}{Monday,\ February\ 12,\ 2016}	% Due date
\newcommand{\hmwkClass}{CPSC\ 525}	 					% Course/class
\newcommand{\hmwkAuthorName}{Christopher Wozniak}		% Your name

%----------------------------------------------------------------------------------------
%	TITLE PAGE
%----------------------------------------------------------------------------------------

\title{
\vspace{2in}
\textmd{\textbf{\hmwkClass:\ \hmwkTitle}}\\
\normalsize\vspace{0.1in}\small{Due\ on\ \hmwkDueDate}\\
\vspace{3in}
}

\author{\textbf{\hmwkAuthorName}}
\date{} % Insert date here if you want it to appear below your name

%----------------------------------------------------------------------------------------

\begin{document}
\maketitle
\newpage

%----------------------------------------------------------------------------------------
%	PROBLEM 1
%----------------------------------------------------------------------------------------
\begin{homeworkProblem}
	We determine first if action taken is morally right, morally wrong, or morally obligatory. Though this is rather simple, as
	the ethical question dictates that you are working for a company whose job is to track botnet activity. As such, you have a
	contract with a governing body to be the authority on botnet activity. Therefore it would be morally obligatory to take action
	against this botnet. However, what type of action should be taken into consideration? Do we simply track the botnet, knowing
	that people (even friends) are being used by the botnet with potentially malicious intent, or do we do everything in our power
	to shut down the botnet, regardless of the job to be performed at work?\\

	We could take the action of the Egoist, seeing the events as an oppertunity for our own gain and silently track the botnet, forwarding
	information up the chain of command and gain favor of bosses and fellow employess for work well done. Deontology would counter
	this by stating that the act of allowing a botnet to run its DDoS activity would be unethical, as the action of watching someone commit
	morally wrong actions is itself morally wrong, regardless of the outcome. Utilitarianism would fit into the argument with a bit of 
	understanding: The act of tracking a botnet may lead to larger utilitarian gains if we are able to track the source. Simply downing the 
	botnet would not stop the opponants from simply starting up another one, and tracking the botnet activity may lead to a greater
	impact than simply stopping the botnet to protect the few slaved users, or targets of a DDoS. Agreeing with Utilitarianism would
	be Social Justice, where the act of a social contract helps protect socitety and allows for such things as the internet to run in the 
	first place. Such a contract allowed there to be a governing body (national, international, corporate law) which authorizes companies
	to track botnet activity. By tracking them, we can better understand who/what is going on and how to stop the source, rather than
	short term botnet activity.\\

	Considering the examples above, the longterm plan of tracking botnet activity, regardless of the impact that a DDoS botnet can have
	is the best action that can be taken as an employee for a security company, whose job is to track such activity. Solutions can be planned
	and information gathered which will lead to a more informed long term solution. Simply shutting down a found botnet would lead to the 
	opponants starting another one back up, only then they have the information that they need to improve their security. Finding the source
	of botnet activity (the users who start it) will provide a greater long term solution which is favorable to Utilitarianism and Social Justice
	theorists, even if deontologists would argue for inaction.

\end{homeworkProblem}

%----------------------------------------------------------------------------------------
%	PROBLEM 2
%----------------------------------------------------------------------------------------
\begin{homeworkProblem}
\begin{enumerate}
	\item 
		\begin{itemize}
			\item \textbf{worm} - Step 7 PLC's were closed off from the internet, so stuxnet had to infect the driver software
				which was used to maintain the PLC's on windows machines. This led to the worm spreading through local networks
				using a printer spooler exploit, and through USB flash drives, as described below.
			\item \textbf{link file} - The 1.001 version of Stuxnet took advantage of an exploit of the Autorun feature
				of a flash drive, where autorun could execute code from a link file found in the flash drive. It would point to the 
				copied stuxnet executable file, and thereby infected the machine it connected to.
			\item \textbf{root kit} - The PLC driver on the windows machine would view memory blocks using a specific .dll, which
				contained system calls to the device. The rootkit hid itself in another dll, stealing the name of the original, and had
				code that emulated the system calls, but hid the virus information from the viewer on the windows device (intentionally
				enumerating over memory blocks on a ``s7blk\_findNext()" call for example.) This dll had to be renamed, and the 
				virus itself had to be placed in the windows registry, which is why the root kit was necessary.
		\end{itemize}
	\item It primarily took advantage of the connection to the PLC device. Managing what the user saw from his embedded interface, it
		made it look as if the device connected to the computer was performing properly. It did this by overwriting the system calls in a 
		specific .dll file (s7otbxdx.dll) with its own version, as mentioned above in the ``root kit" explanation.
	\item The main target of stuxnet was on nuclear capabilities of Iran, attempting to slow down the refinment of nuclear material
		by making hardware components malfunction.
	\item  Duqu and Flame are brothers of Stuxnet, using some of the same source code but primarily used for surveillance. Flame is the 
		stronger of the two, able to take screenshots, track skype activity, capture webcam info and set itself up as a bluetooth base
		station allowing it to spread to wireless devices. Duqu was intended to be a precurser, gathering information on other potential
		targets similar to the PLC's attacked with Stuxnet.
	\item I would argue that stuxnet is a game changer. Before this nobody had attacked embedded systems before, it was also large
		and extremely specific which denotes a virus with international versus simple criminal intent. It opened our eyes to the need
		for computer security beyond our home desktops as the PLC devices were also used to regularly maintain our infrastructure
		in today's modern world. It also was unprecedented in size compared to viruses before its time, showing that viruses could
		be whole applications designed to attack rather than a few lines of corrupted code. There is also the fact that the media 
		actually as interested in the large attack. In and of itself it helped to spread word of a threat that still exists today, with
		our growing dependance of such devices, it could be considered a game changer simply because it brought the attention forward.
	\item Citation
		\begin{itemize}
			\item https://www.symantec.com/content/en/us/enterprise/media/security\_response/whitepapers/w32\_stuxnet\_dossier.pdf
			\item http://www.symantec.com/security\_response/writeup.jsp?docid=2010-071400-3123-99
			\item http://www.symantec.com/connect/blogs/stuxnet-introduces-first-known-rootkit-scada-devices
		\end{itemize}
\end{enumerate}
\end{homeworkProblem}

%----------------------------------------------------------------------------------------
%	PROBLEM 2
%----------------------------------------------------------------------------------------
\newpage
\begin{homeworkProblem}
	The programs are included in the zip, sorted in two different versions.
	\begin{itemize}
		\item \textbf{The Process}
			\begin{enumerate}
				\item \textbf{Read the ELF} and determine what the executable was going to do.
					\begin{lstlisting}
	objdump -S helloworld-x86_64
					\end{lstlisting}
					This gives a list of computed assembler instuctions, which had to be stepped through.
				\item \textbf{Look for hints} since we know what the program is trying to do, I soon found a set of
					instructions which all pointed to one location:
					\begin{lstlisting}
400a64:	e9 04 01 00 00       	jmpq   400b6d <main+0x12d>
...
400b6d:	c9                   	leaveq
					\end{lstlisting}
					With this forsight it was easy to follow the branches down to the set of instructions the guards were avoiding:
					\begin{lstlisting}
  400b59:	bf 7c 0c 40 00       	mov    $0x400c7c,%edi				
  400b5e:	e8 6d fb ff ff       	callq  4006d0 <puts@plt>
					\end{lstlisting}
			\end{enumerate}
		\item \textbf{Version 1} uses ptrace() to change the rip register (x64 program counter) to point to the address containing
			the required code to execute. Its example is in the /programs/ver1\_jump\_over/ folder which includes a README and 
			makefile.
		\item \textbf{Version 2} again uses ptrace(), however it takes the machine code seen below, and overwriting code
			from main onwards so that when the program starts to run it immediately prints the required code. This is found in
			/program/ver2\_overwrite/			
			\begin{lstlisting}
		bf 7c 0c 40 00				
		e8 6d fb ff ff
			\end{lstlisting}
	\end{itemize}

\end{homeworkProblem}
%----------------------------------------------------------------------------------------

\end{document}